
\chapter{Графика} \label{Graphics}

\emph{Иако рачунарска графика на први поглед изгледа једноставна област, што се више задубљујеш у њу, откриваш каква озбиљна наука стоји. }

\section{Општа идеја}
Док су за активности у којима није потребно учестало исцртавања екрана коришћена GUI нит, тамо где је потребно (\emph{CreatePolygonActivity, GameActivity}) морало је бити пронађено другачије решење. За површ уместо стандардног \emph{Canvas}-a који припада \emph{ImageView}, који захтева исцртавање целог екрана \footnote{\url{https://developer.android.com/reference/android/widget/ImageView.html}} користи се \emph{SurfaceView} који се само освежи (остатак екрана се не мења) кад је неопходно. И то исцртавање се ради у посебној нити, која кад заврши посао, само замени \emph{Canvas} од SurfaceView са новим Canvas-ом. Што умањује заузетост GUI нити непотребним исцртавањем, и омогућава да се користи у рачунању и ажурирању SurfaceView.
Ради смањења загревања уређаја, нова нит која исцртава \emph{Canvas} то ради само кад је затражено од ње (кад се десила промена позиције), остатак времена спава. 
\\ \indent
 Цео модел је оптимизован тако да се тражило максималној паралелизацији и минимализацији броја lock-ова. У обе активности се на почетку иницијализације \emph{SurfaceView} правe класе \emph{ShapeFactory} и \emph{ShapeDraw}, од којих прва служи за парсирање полигона из фајлова (и њихово скалирање), док друга служи за цртање фигура по \emph{Canvas}-у \emph{SurfaceView}. Да би фигура била исцртана помоћу класе \emph{ShapeDraw} неопходno je да подржава \emph{ShapeDrawInterface}, тј. да може да се кликне на њу, ротира, промени величина, помери, израчуна угао нагиба. Такође при иницијализацији \emph{SurfaceView} прави се и посебна нит која ће да ради исцртавање. При уништавању \emph{SurfaceView} нит се уништава.
\section{Оптимизације код играња игре}
 Код играња игре, нема потребе за непотребно рендеровање и исцртавање других фигура по \emph{Canvas}-у осим на почетку. Стога се направи спрајт целог полигона без лопте, и лопта се лепи касније на спрајт како се мења њена позиција. Ово омогућава убрзано ажурирање екрана.