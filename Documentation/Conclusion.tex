\chapter{Закључак} \label{Conclusion}

Игрица је за циљ имала учење Android API-а и његову употребу ради оптимизације пројекта. Али проблеми попут рачунарске графике и моделирања физике пребацили су тежиште пројекта.  
\\ \indent Стога игрица је постигла првобитни циљ, а то је Android апликација чији код је читак, добро организован, лак за проширивање. Додавање нових препрека у игри је веома једноставно проширивањем постојећих класа или имплементирањем интерфејса.. Додавање нових опција у модел физике је веома једноставно мењањем једне методе. Механика играња тј. исцртавња фигура апликације је оптимизована до границе кад корисник не осећа да се ради о Android уређају. У суштини, читав апликација је добро организована и проширива у свим погледима. 
\\ \indent Међутим проблем савршене физике остаје отворено питање. Модел који је урађен, довољан је да корисник не примети несавршеност физике у неким деловима. Међутим, требало би да се среди даљим изучавањем моделирања физике у оваквим типовима апликација. 
\\ \indent Следеће отворено питање да ли користити OpenGL и да ли ће он дати још већа побољшања у погледу исцртавања фигура. И са тим питање, да ли треба да се поред лопте која се креће треба да дода неколико покретних препрека. Да ли би оне кориснику дале још више уживања? Све ово захтева темељно испитивање OpenGL API-а за Anrdoid и његово темељно тестирање на одговарајућим моделима физике који буду коришћени.
\\ \indent Треће отворено питање је изглед апликације. Иако је уложен огроман  труд да се постигне да апликација буде што привлачнија за кориснике, видно је да неки делови захтевају побољшање (попут боја препрека, лопте, могућност постављања позадине нивоа).  Стога, треба наћи добро GUI дизајнера за Android који би средио претходно наведене мањкавости. 
\\ \indent Последње отворено питање је сама логика апликације, тј. како се ради adventure мод игрице, и да ли треба дозволити кориснику да прави сам нивое за тај мод. Стога треба извршити истраживање код корисника да се види шта они очекују од овакве апликације по питању нивоа.
\\ \indent И последње питање, да ли апликација треба да иде у продукцију (на Google Play Store линк: \cite{GooglePlayStore}). Ако буде постојала жеља, неопходно је систем тестирања подићи на виши ниво, темељно тестирати све методе и све могуће случајеве коришћења. 