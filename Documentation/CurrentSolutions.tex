\chapter{Преглед решења } \label{CurrentSolutions}

У наредних неколико табела налази се преглед функционалности већ постојећих игара сличних оној која ће бити рађена у дипломском раду (нису све исте по логици, али све имају лопту која се креће), као и жеља за функионалности коју би лопта у дипломском раду подржала.

\begin{table}[H]
\begin{tabular}{|c|c|c|c|c|}
\hline
Игра & Нови полигон & Преглед свих полигона& Рангирање & Звук судара \\
\hline
Rolling Sky\footnote{\url{https://play.google.com/store/apps/details?id=com.turbochilli.rollingsky&hl=sr}} & Не & Да &  Да& Да\\
Physics Drop\footnote{\url{https://play.google.com/store/apps/details?id=com.dreamed.physicsdrop&hl=sr}} & Не & Да &  Не & Не\\
Bouncing Ball\footnote{\url{https://play.google.com/store/apps/details?id=com.theblackeagledev.bouncingball&hl=sr}} & Не & Да &  Не& Не\\
Bounce Classic \footnote{\url{https://play.google.com/store/apps/details?id=com.classic.bounce}} & Не & Да &  Да& Не\\
Rapid Roll \footnote{\url{https://play.google.com/store/apps/details?id=com.nitroxis.rapidroll}} & Не & Не&  Не& Не\\
Balance Ball \footnote{\url{https://play.google.com/store/apps/details?id=com.sumadagames.balanceBall}} & Не & Не&  Да &Не\\
Crazy Balancing Ball\footnote{\url{https://play.google.com/store/apps/details?id=actiongames.games.cbb}} & Не & Да&  Да &Не\\
\hline
\end{tabular}

\caption{Постојећа решења.} \label{ExistingSolutions1}
\end{table}

\begin{table}[H]
\begin{tabular}{|c|c|c|c|c|c|}
\hline
Игра & Трење & Гравитација рупе& Убрзање уређаја& Параметри& Звук игре \\
\hline
Rolling Sky & Не & Не &  Не& Не&Да\\
Physics Drop & Да & Не &  Не & Не&Да\\
Bouncing Ball  & Не & Не&  Не& Не&Да\\
Bounce Classic  & Не & Не &  Не& Не&Да\\
Rapid Roll & Не & Не&  Не& Не&Не\\
Balance Ball  & Не & Не&  Да &Не&Не\\
Crazy Balancing Ball & Да & Не&  Да &Не&Да\\

\hline
\end{tabular}

\caption{Постојећа решења.} \label{ExistingSolutions2}
\end{table}



\begin{table}[H]
\begin{tabular}{|c|c|c|c|c|c|}
\hline
Игра & Круж. препр.& Прав. преп.& Рот. прав. препр.& Вртл. препр.& Одбиj.\\
\hline
Rolling Sky  & Не & Не &  Да&Не&Не\\
Physics Drop  & Да & Да &  Да & Не&Да\\
Bouncing Ball & Не & Да&  Не& Не&Не\\
Bounce Classic & Не & Да &  Да& Не&Не\\
Rapid Roll  & Не & Да&  Не& Не&Не\\
Balance Ball& Да & Да&  Не &Не&Не\\
Crazy Balancing Ball & Да & Да&  Да &Не&Не\\

\hline
\end{tabular}

\caption{Постојећа решења.} \label{ExistingSolutions3}
\end{table}


Као што се може да види из табела \ref{ExistingSolutions1}, \ref{ExistingSolutions2}, \ref{ExistingSolutions3} оно што би одвајало ову игру од постојећих игара било би постојање трења између лопте и подлоге, као и могућност модификовања полигона (уређивања по својој вољи). Поред тога подржавањe свих карактеристика које оне имају у табели (осим звука игре), редом излистаних , као и додатне могућности физике судара.  Књига која ће бити коришћена за упрошћен модел физике је \cite{EngBook}. Поред тога биће коришћен и увид у код који иде уз њу, који се налази на линку адресе у референци \cite{ModCol}.
\\ \indent
За прорамски језик биће коришћена Java, искључиво, јер омогућава лакшу израду пројекта и лакше дебаговање. У комбинацији са Java биће коришћeн xml за израду GUI\footnote{Graphical User Interface}. Биће коришћен  Android Studio\footnote{\url{https://developer.android.com/studio/index.html}} који ће давати лагодност у погледу дебаговања, build-овања и праћења промена. Он омогућава интегрисани рад са системима за праћење ревизија попут Git (линк \cite{BitBucket}). У наставку се налази списак свих алата који ће бити коришћени:
\begin{table}[H]\centering
\begin{tabular}{ l  l } \toprule
{\bf Алат} & {\bf Сврха}\\ \midrule
{\tt AndroidStudio 2.3.3} & IDE\\
{\tt compileSdkVersion 25.0.1} & систем за build-овање уграђен у AndroidStudio\\
{\tt SourceTree 2.1.2.5.} & Систем за ревизију\\
\TeX Maker 4.5 & \LaTeX\ уређивач\\
\bottomrule
\end{tabular}
\caption{Алати коришћени при развоју пројекта.} \label{UsedTools}
\end{table}
Машина на којој ће бити писан и компајлиран код и на којој ће радити Android Studio је HP Omen са 12GB RAM, 512GB SSD, i7-6700HQ 2,7GHz, оперативни систем Windows 10.0.15063. Машина на којој ће бити покретана апликација је NVIDIA Shield Tablet K1, који је у тренутку првог инсталирања имао Android 5.0 верзију инсталирану. \footnote{У тренутку писања последњих измена ажуриран је на Android 7.0}. Минимални Андроид ОС\footnote{Оперативни систем} који подржава апликацију је 5.0.
