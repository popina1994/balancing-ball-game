
\chapter{Увод}
Основни циљ пројекта био је истраживање Android API\footnote{Application programming interface}-а који се користи за прављење апликација за Android уређаје. Да би се у потпуности разумела његова употреба и организација кода у апликацијама намењеним Android-у, морало се приступити изазову прављења игре попут ове. Поред Android API-а, због одабира игре у којој је потребна симулација лопта, постојали су проблеми моделирања физике и рачунарске графике.
\\ \indent Први у низу проблема који је решаван био је делимично позавање API-а. Први корак ка његовом превазилажењу био је предмет програмирање мобилних уређаја (\cite{PMU}), на којем су научене основне ствари које он нуди. Други корак је био читање документације на AndroidDeveloper сајта (\cite{AndroidDeveloper}) на ком се осим описа пакета, класа, интерфејса, метода налазе савети и упутства за коришћење Android API-а. Последњи корак на који се прибегавало кад проблем није могао бити решен је StackOverflow (\cite{StackOverflow}). Наведена секвенца у процесу решавања омогућила је превазилажење препрека и давала мотивацију за даљи рад у борби са непознатим.
\\ \indent Пошто је постојала жеља за што реалнију симулацију лопте, и њене физика приликом њеног динамичког, а и статичког кретања, појавила су се два нова проблема. Први од њих био је рачунарска графика, чија мотивација  потиче из жеља аутора да наликује графици у познатим играма попут \emph{GTA} (линк \cite{GTA}). Први корак ка томе било је моделирање нечега што се може назвати модел физике. Он ће омогућити кориснику видно савршену симулацију физичких односа између објеката у игри. Он је уједно и био други проблем. Међутим дати модел је требало поткрепити осећајем да не постоји освежавање екрана. Нити да корисник има могућност да разликује лопту у стварности од оне у игрици. Стога је графика оптимизована у виду паралелизације. А физика моделирана по угледу на физику у стварном свету.
\\ \indent Последњи изазов, ако не и најважнији била је организација кода у великом пројекту као што је овај која омогућава брзу проширивост и лагано додавање нових функционалнсти, мењање постојећих модела судара... Овај проблем, иако на први поглед наиван, како се повећавала количина кода био је све већи. Стога се код у одређеним фазама рефакторисао и разбијао на модуле. Крајњи код је последица жеље аутора да се добије модуларан и проширив код.
\\ \indent
У глави \ref{CurrentSolutions} биће наведене функционалности које сличне игре подржавају на тржишту. 
У глави  \ref{GameProblem} биће наведене функционалности које би игра требала да подржава. 
У глави \ref{ModelProposal} биће описана игра, тј. како би требало оквирно да изгледа и шта ће који екран да подржи. Осим тога биће наведено и развојно окружење у којем је игра рађена. Поред тога биће наведен модел физике и модел графике који ће бити коришћен. У глави \ref{Impl} биће наведена имплементација игре у погледу кода графичког интерфејса, Java кода, чувања података, тестирања као и архитектуре самог кода. 
У глави \ref{UseCases} биће описани случајеви коришћења система и како се систем понаша у одређеним тренуцима. У глави \ref{Conclusion} биће наведени закључци, тачније шта је био циљ, шта је урађено и предолзи за даљи рад. 





