
\chapter{Увод}
\emph{Основни циљ пројекта био је да истражим Android API\footnote{Application programming interface} који се користи за прављење апликација за Android уређаје. Да бих у потпуности разумео његову употребу и организацију кода у апликацијама за Android, морао сам да приступим изазову прављења игрице попут ове. Поред Android API-а први пут сам се сусрео са проблемима моделирања физике у рачунарској графици. У овом поглављу сам се осврнуо на мотивацију за имплементацију на овај начин, као и изазове са којима сам се сусрео у изради пројекта. }
\section{Мотивација}
Вештина коју програмер мора да има је прилагођавање новонасталим ситуацијама и решавање проблема који дођу у њима. 
\\ \indent Прављење великог пројекта са  непознатим API-ем један од великих изазова на које сам наишао. Први корак у томе био је предмет програмирање мобилних уређаја (\cite{PMU}), који сам слушао и на ком сам научио основне ствари које нуди Android API. Други корак је било злостављање AndroidDeveloper сајта (\cite{AndroidDeveloper})на ком смо налазили на предмету одговарајуће класе, методе, константе, чланке који су нам омогућавали прављење жељених апликација. Последњи корак на који се прибегавало кад проблем није могао бити решен је StackOverflow (\cite{StackOverflow}). Наведени кораци су ми омогућавали и давали мотивацију за даљи рад у борби са непознатим.
\\ \indent Други велики мотив био је рачунарска графика. Пошто као сваки мали дечак у великом телу сам имао жељу да научим како раде игрице попут GTA (\cite{GTA}). Први корак ка томе било је моделирање нечега што се може назвати модел судара. Дати пројекат је био изазован у погледу организације кода за модел судара, јер су сви делови уско спрегнути.
\\ \indent Последњи мотив, а можда и најважнији била је организација кода у великом пројекту као што је овај, која омогућава брзу проширивост и лагано додавање нових функционалнсти, мењање постојећих модела судара... Ово је врхунац знања сваког програмера.

\section{Изазови}

	