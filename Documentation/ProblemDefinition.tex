\chapter{Поставка проблема}\label{GameProblem}

Идеја је да се направи игра, која нуди скоро све што и конкурентне игре које постоје на тржишту и које се баве истом тематиком. Али поред тога да понуди  додатне функционалности које друге игре немају. Као и модел физике који корисник неће моћи да разликује од стварности. Поред тога нудила би брзу графику која ће омогућити појаву огромног броја објеката на екрану без губљења осећаја стварности. Следи освртај на главу \ref{CurrentSolutions} и шта нуде постојеће игре у виду функционалности.
\\ \indent
 Из табеле \ref{ExistingSolutions1} закључује се да ниједна од њих не омогућава кориснику да сам прави своје полигоне (нивое). Омогућавање такве функционалности даће кориснику жељу да не игра друге игре, него баш ову. Следећа ставка је преглед полигона, њу нуди скоро свака игра што се види из исте табеле. Међутим, ни у једној од њих корисник нема могућност прегледа нивоа пре него што почне, што би кориснику сачувало неколико секунди времена, ако му се не свиђа полигон. Тако да поред прегледа свих постојећих полигона, требало би додати и могућност, да на неки начин, без играња игре корисник види полигон који ће играти. Следећа ставка је рангирање по неком основу. Из исте табеле закључујемо да неке игре имају неке немају, али у случају ове игре биће омогућено да се корисници рангирају по полигонима на одређени начин. Последња из ове табела је звук судара лопте и препреке. Из табеле се види да ретко која игра подржава ту функционалност, а она би кориснику дала стварнији осећај, јер корисник обично сударе у стварности примећује по звуку.
\\ \indent
Из табеле \ref{ExistingSolutions2} закључује се да неке игре подржавају трење између лопте и подлоге. А у оваквим играма увек је боље кад лопта може да се заустави, јер корисници ће тако лакше прелазити полигоне и желеће више да играју игру. Следећа ставка из табеле је гравитација рупе, под овим се мисли крајње рупе у коју лопта треба да уђе, као и рупа препрека. Као што може да се види, ниједна од игара је не подржава, тако да би корисник запамтио игру и по тој функционалности. Следећа могућност је реаговање лопте на убрзање уређаја (као и силу гравитације). Из табеле се види да је неке подржавају, неке не, али ако корисник треба да има осећај да држи лопту на нечему, ово је сигурно једна од кључних ставки. Оно што друге игре немају је параметризовање коефицијената попут трења, убзрања, губљења енергије, а корисници који воле да праве нивое сигурно воле и да експериментишу са физикама. Стога, и она би била пожељна. Последња ставка у овој табели је звук игре. Мада већина игара подржава, за сад не би било паметно додавати је, јер корисник би изгубио осећај да се ради о балансирању лопте.
\\ \indent
Из табеле \ref{ExistingSolutions3} уочавају се препреке које су подржане у играма као и да ли је могуће одбијање. За кружну ставку је урачунате су препреке које су у 2D крућне и омогућавају лопте одбијање од њих или упадање у њих. Доста игара је подржава, тако да би требало подржати. Правоугаона препрека је скоро у свима могућа, док је ротирана правоугаона препрека у некима, па би било пожељно подржати. Вртложна препрека, која ће се понашати као вртлог воде и привлачити лопту ка центру не постоји ни у једној игри, и дала би препознатљиву препреку за игру. Последња ставка, мало зачуђујуће, је одбијање, која је подржана само у једној од игара. Почетна замисао игре од почетка подржава одбијање лопте од било чега, тако да ће омогућити велику предност у односу на конкуренцију.
