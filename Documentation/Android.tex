\chapter{Android}\label{Android}

\emph{Још од избацивања IPhone-а као првог комерцијалног "паметног" телефона који ће објединити у једном уређају све функционалности  које има PC (линк \cite{IPHONE}), ако не и више, било је јасно у ком смеру се запутило IT \footnote{ Internet Technology} тржиште. Телефони или ти компјутери у џепу су у том тренутку постали будућност, а наша садашњост. Како ће се развијати IT технологија, није познато, али зна се да ће огроман удео имати телефони. Одатле и потиче моја  жеља да овладам вештином Android програмирања, оперативним системом који је преузео примат на тржишут паметних телефона. (линк \cite{MarkShare}) }
\section{GUI код}
Сав GUI који је коришћен, писан је у xml-у који подржва одговарајуће Android Studio IDE. Сва имена GUI компоненти давана су тако да прво иде тип компоненте и затим се надовезује одговарајући опис који је карактеристичан за употребу дате компоненте. Типа \emph{seekBarCoeficientAcceleration} означава \emph{seekBar} GUI компоненту, а CoeficientAcceleration каже да се користи да представи коефицијент убрзања. Имена сва су писана CamelCase-ом \footnote{\url{http://wiki.c2.com/?CamelCase}}. При избору компоненти биране су тако да што пријатније изгледа кориснику и да што лагодније буде за рад (на основу узорка од неколико корисника који су пробали различите верзије GUI-а). 
\\ \indent 
Где је било неопходно да позиционирање компоненти буде независно од типа од екрана коришћен је \emph{LinearLayout}
\footnote{\url{https://developer.android.com/reference/android/widget/LinearLayout.html}}, 
док је за неке ствари где је битна само позиција компоненти, коришћен \emph{RelativeLayout}\footnote{\url{https://developer.android.com/guide/topics/ui/layout/relative.html}}.
\\ \indent 
Коришћена је Android Dark Material Theme као основна тема.

\section{Java код}
GUI компоненте на одговарајућим екранима референциране су тако што се из одговарајућег прозора нађе компонента уз помоћ \emph{findViewById} \footnote{\url{https://developer.android.com/reference/android/view/View.html} и \url{https://developer.android.com/reference/android/app/Activity.html}}
\\ \indent 
Постоји пет активности (\emph{MainActivity, CreatePolygonActivity, GameActivity, SettingsActivity, StatisticsActivity}). Свака подржавајући одговарајућу функционалност из главе \ref{UseCases}. Дакле MainActivity почетни екран (\ref{UseCases:Main}), GameActivity  саму игру (\ref{UseCases:Game}), SettingsActivity подешавања (\ref{UseCases:Settings}), StatisticsActivity резултате (\ref{UseCases:Statistics}) и CreatePolygonActivity уређивање полигона (\ref{UseCases:CreatePolygon}).
\\ \indent 
Свака активност је прављена тако да је изведена из активности \emph{CommonActivtity}. Свака активност која наслеђује ову класу има подешен прозор тако да је навигациона трака скривена док корисник не превуче прстом са дна уређаја на горе. Поред тога оријентација је увек вертикално, да корисник не би губио време ако случајно окрене уређај. Такoђе екран обавештења је остао доступан кориснику да би могао у сваком тренутку да сазна више о обавештењу које му стигне (али тек након што превуче прстом екран са врха ка дну).
 Дата имплементације је по узору на већину данашањих екслузивних играчких наслова за Android уређаје попут Hill Climb Racing (линк за преузимање \cite{HillCR}). Даље коришћен је MVC\footnote{Model View Controller} пројектни узорак прилагођен за Android. При чему имамо активност која прослеђује своје догађаје контролеру, и он у зависности од њих обавља акције и у моделу се то чува. Постоји и имплементација где у моделу постоје методе које обрађују податке, али изабраним је раздвојена имплементација кода , од приступа подацима, и олакшана читкоћа кода.  Тамо где није био велики обим потребних метода (MainActivity, StatisticsActivity и SettingsActivity) обједињени су Controller и View. 
 \\ \indent
 Постојање класе \emph{CommonModel} за циљ има омогућавање заједничког модела свим активностима које су за потребу имали рендеровање направљеног/који се прави полигона. 
 \\ \indent
 Тамо где је разумно било да се појављују дијалози (као за чување резултата по успешној игри, или за чување направљеног полигона) прављене су класе (\emph{SaveDialog} и \emph{GameOverDialog}) које проширују класу 
 \emph{Dialog}\footnote{\url{https://developer.android.com/reference/android/app/Dialog.html}} и праве одговарајући потребан GUI. 
 Ово је омогућило финије подешавање дијалога и њиховог изгледа oд унапред направљених класа попут \emph{AlertDialog}\footnote{\url{https://developer.android.com/reference/android/app/AlertDialog.html}}.
\\ \indent 
Тамо где су била потребна исцртавања на екрану (\emph{CreatePolygonActivity} и \emph{GameActivity}) коришћене су одговарајућe класе које су проширивале \emph{SurfaceView} \footnote{\url{https://developer.android.com/reference/android/view/SurfaceView.html}} (биће касније објашњено како ово ради у глави \ref{Graphics}). 
\\ \indent 
За \emph{SettingsActivity} било је потребно  унапредити постојећу GUI компоненту SeekBar. Због тога је додата класа \emph{SeekBarUpgrade} која додаје још неки низ особина и омогућава лакше додавање више SeekBar-ова у SettingsActivity. 
\\ \indent 
Пошто је било неопходно да лопта реагује на силу која делује на Android уређај у \emph{GameActivity} активности , коришћен је уграђени Android сензор 
\emph{TYPE\_ACCELEROMETER}\footnote{\url{https://developer.android.com/guide/topics/sensors/sensors_overview.html}}.
Он сваких \si{20ms} прослеђује активности  детектоване вредности убрзања уређаја по \emph{x}, \emph{y} и \emph{z} оси. Њихово коришћење је описано даље у глави \ref{Collision}. Oва вредност од \si{20ms} емпиријским утврђивањем се показала као довољна да корисник стекне осећај реалистичности кретања куглице и реаговања исте на силе које делују на уређај. 
\\ \indent 
Да би корисник стекао што реалистичнији осећај кретања лоптице и тренутка њеног судара са препрекама или уласка у њих неопходно је било подржати звук. Изабрана је класа \emph{SoundPool}\footnote{\url{https://developer.android.com/reference/android/media/SoundPool.html}} која омогућава пуштање звукова. Имплемнтиран је омотач за њу у виду класе \emph{SoundPlayer} који додаје још неке функционалности. Звучни ефекти су одабрани емпиријски да што краће трају и да корисник не обража превелику пажњу на њих. Детаљније у методама, интерфејсу и свим класама за звукове у глави \ref{Architecture}.

\section{Чување података}
Постојале су потребе за три начина чувања података у апликацији. 
\\ \indent 
Први потребан начин је било перзистирање коефицијената потребних за симулацију кретања лопте. Пошто су они били потребн да перзистирају дуж активности \emph{GameActivity} и \emph{SettingsActivity} и није их било пуно, одбачене су опције да се чувају у бази и посебном фајлу. Стога је одабрана опција \emph{SharedPreferences}\footnote{\url{https://developer.android.com/reference/android/content/SharedPreferences.html}} која чува податке у облику пара key/value. Ово омогућава згодно додавање нове константе без гломазних мењања база и без мењања шеме фајла. Овај систем је имплементиран у класи \emph{Coefficient}.
\\ \indent 
Други потребан начин било је перзистирање података везаних за име нивоа, тежину, као и рангирање нивоа. То је омогућено коришћењем SQLite језика који је ништа друго него лакша варијанта SQL\footnote{Standard Query Language} језика за рад са базом. Прво је имплементиран \emph{SQLiteOpenHelper} у виду класе \emph{GameDatabaseHelper} који омогућава мењање шеме базе (ако се врши додавање нове функционалности у игрици везане за нивое), као и приступ истој. Даље је \emph{GameDatabaseHelper} омотан у класи \emph{GameDatabase} која имплементира читав низ функционалности који је био неопходан у апликацији. 
\\ \indent 
Последњи потребан начин било је перзистирање нивоа (њиховог изгледа). Одабран је начин чувања у фајловима. То омогућава згодно додавање нових фигура само додавањем новог типа фигуре, и не изискује гломазно мењање базе података и API-а функционалности базе. Пошто је било неопходно да се омогући перзистирање димензија фигура на полигону за сваки тип уређаја, то је урађено скалирањем фигура у односу на величину екрана. Скалирање фигура се обавља уз помоћ класa \emph{UtilScale} и \emph{UtilScaleNormal} (апстрактна и имплементација). Класа која чува полигоне у облику фајлова у овом формату је \emph{ShapeParser}. 

\section{Unit Test}
За потребе тестирања коришћени су корисници који су играли ову игрицу, као и \emph{JUnit} пакет који је омогућио тестирање одређених метода класа и функционалности. Да би ова апликација ишла у продукцију, неопходна су бројна унапређења у тестирању као и броју тестова.