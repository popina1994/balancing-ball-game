\chapter{Android}\label{Android}

\emph{Још од избацивања IPhone-а као првог комерцијалног "паметног" телефона који ће објединити у једном уређају све функционалности  које има PC (линк \cite{IPHONE}), ако не и више, било је јасно у ком смеру се запутило IT \footnote{ Technoloy} тржиште. Телефони или ти компјутери у џепу су у том тренутку постали будућност, а наша садашњост. Како ће се развијати IT технологија, није познато, али зна се да ће огроман удео имати телефони. Одатле и потиче моја  жеља да овладам вештином Android програмирања, оперативним системом који је преузео примат на тржишут паметних телефона. (линк \cite{MarkShare}) }
\section{GUI код}
Сав GUI који је коришћен, писан је у xml-у који подржва одговарајуће Android Studio IDE. Сва имена GUI компоненти давана су тако да прво иде тип компоненте и затим се надовезује одговарајући опис који је карактеристичан за употребу дате компоненте. Типа \emph{seekBarCoeficientAcceleration} означава \emph{seekBar} GUI компоненту, а CoeficientAcceleration каже да се користи да представи коефицијент убрзања. Имена сва су писана CamelCase-ом \footnote{\url{http://wiki.c2.com/?CamelCase}}. При избору компоненти биране су тако да што пријатније изгледа кориснику и да што лагодније буде за рад (на основу узорка од неколико корисника који су пробали различите верзије GUI-а). 
\\ \indent 
Где је било неопходно да позиционирање компоненти буде независно од типа од екрана коришћен је \emph{LinearLayout}
\footnote{\url{https://developer.android.com/reference/android/widget/LinearLayout.html}}, 
док је за неке ствари где је битна само позиција компоненти, коришћен \emph{RelativeLayout}\footnote{\url{https://developer.android.com/guide/topics/ui/layout/relative.html}}.
\\ \indent 
Коришћена је Android Dark Material Theme као основна тема.

\section{Java код}
GUI компоненте на одговарајућим екранима референциране су тако што се из одговарајућег прозора нађе компонента уз помоћ \emph{findViewById} \footnote{\url{https://developer.android.com/reference/android/view/View.html} и \url{https://developer.android.com/reference/android/app/Activity.html}}
\\ \indent 
Постоји пет активности (\emph{MainActivity, CreatePolygonActivity, GameActivity, SettingsActivity, StatisticsActivity}). Свака подржавајући одговарајућу функционалност из главе \ref{UseCases}.
\\ \indent 
Свака активност је прављена тако да је изведена из активности \emph{CommonActivtity}. Свака активност која наслеђује ову класу има подешен прозор тако да је навигациона трака скривена док корисник не превуче прстом са дна уређаја на горе. Поред тога оријентација је увек вертикално, да корисник не би губио време ако случајно окрене уређај. Такoђе екран обавештења је остао доступан кориснику да би могао у сваком тренутку да сазна више о обавештењу које му стигне (али тек након што превуче прстом екран са врха ка дну).
 Дата имплементације је по узору на већину данашањих екслузивних играчких наслова за Android уређаје попут Hill Climb Racing (линк за преузимање \cite{HillCR}). Даље коришћен је MVC пројектни узорак прилагођен за Android. При чему имамо активност која прослеђује своје догађаје контролеру, и он у зависности од њих обавља акције и у моделу се то чува. Постоји и имплементација где у моделу постоје методе које обрађују податке, али изабраним је раздвојена имплементација кода , од приступа подацима, и олакшана читкоћа кода. 
 \\ \indent
 Постојање класе \emph{CommonModel} за циљ има омогућавање заједничког модела свим активностима које су за потребу имали рендеровање направљеног/који се прави полигона. 
 \\ \indent
 Тамо где је разумно било да се појављују дијалози (као за чување резултата по успешној игри, или за чување направљеног полигона) прављене су класе које проширују класу 
 \emph{Dialog}\footnote{\url{https://developer.android.com/reference/android/app/Dialog.html}}. 
 Ово је омогућило финије подешавање дијалога и њиховог изгледа oд унапред направљених класа попут \emph{AlertDialog}\footnote{\url{https://developer.android.com/reference/android/app/AlertDialog.html}}.
\\ \indent 
Тамо где су била потребна исцртавања на екрану коришћене су одговарајућe класе које су проширивале \emph{SurfaceView} \footnote{\url{https://developer.android.com/reference/android/view/SurfaceView.html}} (биће касније објашњено како ово ради у глави \ref{Graphics}). 
\\ \indent 
За \emph{SettingsActivity} било је неопходно унапредити постојећу GUI компоненту SeekBar и додата је класа \emph{SeekBarUpgrade} која додаје још неки низ особина и омогућава лакше додавање више SeekBar-ова.

\section{Unit Test}