\documentclass[hidelinks,a4paper,12pt]{report}

\usepackage[utf8x]{inputenc}
\usepackage[T2A]{fontenc} 
\usepackage[serbianc]{babel}
\usepackage[colorlinks = true,urlcolor = blue]{hyperref}
\usepackage{setspace, amsmath, amsfonts, amssymb, amsthm, enumerate, graphicx, color}
\usepackage[left=1in, right=1.0in, top=1.0in, bottom=1.0in]{geometry}
\usepackage{listings}
\usepackage{algorithm}
\usepackage{algpseudocode}
\usepackage{booktabs}

\newcommand{\HRule}{\rule{\linewidth}{0.5mm}}
\newcommand{\conj}{\overline}
\newcommand{\trig}{r(\cos\varphi+i\sin\varphi)}	
\newcommand{\trign}{r^n(\cos(n\varphi)+i\sin(n\varphi))}
\newcommand{\trigj}{r(\cos\alpha+i\sin\alpha)}
\renewcommand{\vec}{\overrightarrow}

%% Custom colours
\definecolor{darkgreen}{rgb}{0.0, 0.42, 0.24}
\definecolor{light-gray}{gray}{0.85}
% C++ Code%00
\definecolor{mygreen}{rgb}{0,0.6,0}
\definecolor{mygray}{rgb}{0.5,0.5,0.5}
\definecolor{mymauve}{rgb}{0.58,0,0.82}

\lstset{ %
  backgroundcolor=\color{white},   % choose the background color; you must add \usepackage{color} or \usepackage{xcolor}
  basicstyle=\footnotesize\ttfamily,        % the size of the fonts that are used for the code
  %belowskip=-0.8\baselineskip,
  breakatwhitespace=false,         % sets if automatic breaks should only happen at whitespace
  breaklines=true,                 % sets automatic line breaking
  captionpos=b,                    % sets the caption-position to bottom
  commentstyle=\color{mygreen},    % comment style
  deletekeywords={...},            % if you want to delete keywords from the given language
  escapeinside={\%*}{*)},          % if you want to add LaTeX within your code
  extendedchars=true,              % lets you use non-ASCII characters; for 8-bits encodings only, does not work with UTF-8
  frame=false,                    % adds a frame around the code
  keepspaces=true,                 % keeps spaces in text, useful for keeping indentation of code (possibly needs columns=flexible)
  keywordstyle=\color{blue},       % keyword style
  language=C++,                 % the language of the code
  literate={~} {\mytilde}{1},
  morekeywords={*,...},            % if you want to add more keywords to the set1
  numbers=left,                    % where to put the line-numbers; possible values are (none, left, right)
  numbersep=5pt,                   % how far the line-numbers are from the code
  numberstyle=\tiny\color{mygray}, % the style that is used for the line-numbers
  rulecolor=\color{black},         % if not set, the frame-color may be changed on line-breaks within not-black text (e.g. comments (green here))
  showspaces=false,                % show spaces everywhere adding particular underscores; it overrides 'showstringspaces'
  showstringspaces=false,          % underline spaces within strings only
  showtabs=false,                  % show tabs within strings adding particular underscores
  stepnumber=1,                    % the step between two line-numbers. If it's 1, each line will be numbered
  stringstyle=\color{mymauve},     % string literal style
  tabsize=2,                       % sets default tabsize to 2 spaces
  title=\lstname                   % show the filename of files included with \lstinputlisting; also try caption instead of title
}


\raggedbottom                           % try to avoid widows and orphans
\sloppy
\clubpenalty1000%
\widowpenalty1000%

\renewcommand{\baselinestretch}{1.1}    % adjust line spacing to make
                                        % more readable


\begin{document}




\begin{titlepage}

\begin{center}


% Upper part of the page   

\textsc{\Huge \textbf{Универзитет у Београду}} \\[1cm]
\textsc{\Huge \textbf{Електротехнички факултет}}\\[5cm]

\textsc{\Huge \textbf{Дипломски рад }}\\[0.5cm]
\textsc{\large{\textbf{из програмирања мобилних уређаја}}}
\\[4cm]

% Title
\HRule \\[0.8cm]
{ \Huge \bfseries Android апликација - лопта на плочи}\\[0.4cm]

\HRule \\[5cm]

% Author and supervisor
\begin{minipage}{0.55\textwidth}
\begin{flushleft} \large
\emph{Аутор}\\
Ђорђе Живановић, 0033/2013\\
\end{flushleft}
\end{minipage}
\begin{minipage}{0.4\textwidth}
\begin{flushright} \large
\emph{Ментор} \\
Др Саша Стојановић
\end{flushright}
\end{minipage}



\vfill

% Bottom of the page
{\large Београд, \today}

\end{center}

\end{titlepage}

\tableofcontents

\renewcommand{\bibname}{Литература и линкови}
\pagenumbering{arabic}
\renewcommand{\chaptername}{ Глава}


\chapter{Увод}
\emph{Основни циљ пројекта био је да истражим Android API\footnote{Application programming interface} који се користи за прављење апликација за Android уређаје. Да бих у потпуности разумео његову употребу и организацију кода у апликацијама за Android, морао сам да приступим изазову прављења игрице попут ове. Поред Android API-а први пут сам се сусрео са проблемима моделирања физике у рачунарској графици. У овом поглављу сам се осврнуо на мотивацију за имплементацију на овај начин, као и изазове са којима сам се сусрео у изради пројекта. }
\section{Мотивација}
Вештина коју програмер мора да има је прилагођавање новонасталим ситуацијама и решавање проблема који дођу у њима. 
\\ \indent Прављење великог пројекта са  непознатим API-ем један од великих изазова на које сам наишао. Први корак у томе био је предмет програмирање мобилних уређаја (\cite{PMU}), који сам слушао и на ком сам научио основне ствари које нуди Android API. Други корак је било злостављање AndroidDeveloper сајта (\cite{AndroidDeveloper})на ком смо налазили на предмету одговарајуће класе, методе, константе, чланке који су нам омогућавали прављење жељених апликација. Последњи корак на који се прибегавало кад проблем није могао бити решен је StackOverflow (\cite{StackOverflow}). Наведени кораци су ми омогућавали и давали мотивацију за даљи рад у борби са непознатим.
\\ \indent Други велики мотив био је рачунарска графика. Пошто као сваки мали дечак у великом телу сам имао жељу да научим како раде игрице попут GTA (\cite{GTA}). Први корак ка томе било је моделирање нечега што се може назвати модел судара. Дати пројекат је био изазован у погледу организације кода за модел судара, јер су сви делови уско спрегнути.
\\ \indent Последњи мотив, а можда и најважнији била је организација кода у великом пројекту као што је овај, која омогућава брзу проширивост и лагано додавање нових функционалнсти, мењање постојећих модела судара... Ово је врхунац знања сваког програмера.

\section{Постојећа решења}
Битна особина коју програмер мора да поседује је способност да чита туђи код и да прилагођава својој имплементацији. Књига која ми је је помогла да направим модел судара какав је у игрици је \cite{EngBook}. Она ми је омогућила да добијем увид у организацију кода и она је само била увод у моје решење. Поред тога модификовани су разни модели и тестирани за потребе. За доста реалнији модел постоји пуно више радова на интернету, али за почетну идеју користио сам упрошћену имплементацију једног модела из књиге \cite{EngBook}, која се налази у линку: 		\cite{ModCol}.

\section{Коришћени алати}
Користио сам прорамски језик Java искључиво јер омогућава лакшу израду пројекта и лакше дебаговање. У комбинацији са Java коришћeн је xml за израду GUI\footnote{Graphical User Interface}.Коришћен је Android Studio\footnote{\url{https://developer.android.com/studio/index.html}} који ми је дао лагодност у погледу дебаговања, build-овања и праћења промена. Он омогућава интегрисани рад са системима за праћење ревизија попут Git. У наставку се налази списак свих коришћених алата:
\begin{table}[H]\centering
\begin{tabular}{ l  l } \toprule
{\bf Алат} & {\bf Сврха}\\ \midrule
{\tt AndroidStudio 2.3.3} & IDE\\
{\tt compileSdkVersion 25.0.1} & систем за build-овање уграђен у AndroidStudio\\
{\tt SourceTree 2.1.2.5.} & Систем за ревизију\\
\TeX Maker 4.5 & \LaTeX\ уређивач\\
\bottomrule
\end{tabular}
\caption{Алати коришћени при развоју пројекта.} \label{UsedTools}
\end{table}


\section{Структура}
У глави  \ref{Collision} биће причано о моделу који судара који је коришћен и зашто су биране његове компоненте. У глави \ref{Graphics} биће описан систем за исцртавање екрана. У глави \ref{Architecture} биће описана софтверска архитектура (пројектни обрасци, организација кода). У глави \ref{UseCases} биће наведени неки случајеви коришћења система и како се систем понаша у одређеним тренуцима. У глави \ref{Conclusion} биће наведени закључци, да ли систем може да се побоља, да ли могу да се додају нове функционалности и сл.


	
\chapter{Функционалности} \label{UseCases}
\emph{Ма колико изгледало наивно и просто направити дизајн који ће привући корисника, док не кренете да правите апликацију, не схватите колико немате појма шта просечан корисник хоће. Стога након пажљивог разматрања направљено је пет "екрана" из којих кориснк приступа одговарајућим функционалностима које ће бити изложене у даљем тексту}

\section{Почетни екран} \label{UseCases:Main}

\subsection{Основни}
\begin{figure}[htb!]
\begin{center}
\includegraphics[scale=.1]{pictures/main/Basic}
\caption{Почетни екран}\label{fig:mainBasic}
\end{center}
\end{figure}
Кад корисник покрене апликацију појављује му се листа полигона које може да игра, са изгледом полигона (слика \ref{fig:mainBasic}). Такође изнад изгледа полигона налазе се његово име и тежина. Ово омогућава да кориснику почетнику одабере ниво прикладан за њега, или експерту да се окуша са нечим тежим. Изглед полигона је скалиран да одговара оном који ће бити у игрици. Кликом на било који од полигона у листи покреће му се игра за тај полигон.

\subsection{Meни} 
\begin{figure}[htb!]
\begin{center}
\includegraphics[scale=.1]{pictures/main/Menu}
\caption{Мени}\label{fig:mainMenu}
\end{center}
\end{figure}
Корисник све време на врху прозора има мени који отвара ако прстом кликне на три тачке. Из менија који се појави (слика \ref{fig:mainMenu}) корисник може да одабере једну од 4 опције. Прва опција започиње нову игру, али у режиму од више нивоа који су насумично генерисани. Друга опција отвара нови екран у ком корисник прави свој полигон. Трећа опција отвара нови екран резултати, где корисник може да види све резултате претходних игри, као и других играча. Четврта опција отвара подешавања за игру, која корисник може да мења.


\subsection{Опције полигона}
\begin{figure}[htb!]
\begin{center}
\includegraphics[scale=.1]{pictures/main/PolygonMenu}
\caption{Падајући мени везан за полигон}\label{fig:mainPolygonMenu}
\end{center}
\end{figure}
Корисник задржавањем прста на полигон добија падајући мени (слика \ref{fig:mainPolygonMenu}) са две опције. Прва опција брише полигон из целе игре. А друга омогућава његово уређивање, тако што отвара нови прозор са датим полигоном.

\section{Игра}\label{UseCases:Game}
\begin{figure}[htb!]
\begin{center}
\includegraphics[scale=.1]{pictures/game/game}
\caption{Тренутак у игри}\label{fig:gameGame}
\end{center}
\end{figure}


Постоје два мода играња. Први мод је обичан у ком је циљ да црвену лопту корисник убаци у зелену рупу за што краће време (слика \ref{fig:gameGame}). Што му је време краће, биће боље рангиран на листи за тај полигон. Други мод је авантуристчки који омогућава кориснику да игра десет насумичних нивоа заредом који су поређани по растућој тежини и биће рангиран на посебној листи  за тај мод. Корисник има неколико типова препрека на које може да наиђе (слика \ref{fig:createPolygonBasic}). Типови препрека:
\begin{enumerate}
	\item Зид (ивица екрана и жути правоугаоник) - лопта се одбија од зида под истим углом којим улази. Губитак енергије током судара зависи од коефицијента подешеног у подешавањима.
	\item Стуб (жути круг) - исто као зид, са тим да је кружног облика.
	\item Амбис (црни круг) - лопта кад упадне у њега играч губи игру.
	\item Вртлог (плави круг) - покушава да увуче лопту у њега, и потом у зависности од брзине баца из њега или га врти.
	\item Крај (зелени круг) - кад лопта упадне у њега корисник је победио дати полигон.\footnote{Амбис, вртлог и крај кад лопта их дотакне вуку је ка себи одређеном силом}
\end{enumerate}
Лопта све време добија брзину у зависности од силе која делује на Android уређаја. На лопту у сваком тренутку делује трење од стране подлоге које покушава да је врати у стање мировања. У сваком тренутку корисник може да изађе из одговарајућег полигона, али неће му бити сачувано ништа што је играо.
\subsection{Игра побеђена}
\begin{figure}[htb!]
\begin{center}
\includegraphics[scale=.1]{pictures/game/gameWon}
\caption{Дијалог кад победи}\label{fig:gameGameWon}
\end{center}
\end{figure}


\begin{figure}[htb!]
\begin{center}
\includegraphics[scale=.1]{pictures/game/gameHighScore}
\caption{Листа резултата за полигон који је победио}\label{fig:gameGameHighScore}
\end{center}
\end{figure}

Кад корисник убаци лопту у зелену рупу, искаче му дијалог (слика \ref{fig:gameGameWon}). На дијалогу пише време, као и нуди кориснику да унесе своје име. Кад кликне на дугме \emph{SACUVAJ REZULTAT} корисников резултат се чува у листи резултата за тај плигон и отвара листа резултата за исти(слика \ref{fig:gameGameHighScore}).
\subsection{Игра изгубљена}
\begin{figure}[htb!]
\begin{center}
\includegraphics[scale=.4]{pictures/game/gameLose}
\caption{Обавештење кад корисник изгуби}\label{fig:gameLose}
\end{center}
\end{figure}
Када корисник упадне у амбис изађе му обавештење да је изгубио игру (слика \ref{fig:gameLose}). 

\section{Прављење полигона}\label{UseCases:CreatePolygon}
Постоје два мода у која корисник може да уђе кад уређује полигон. 
\\ \indent Један је мод уређивања, у ком се налази тако што или је кликнуо \emph{Uredi poligon} на почетном екрану или је сачувао полигон под неким именом. У овом моду кад корисник кликне назад аутоматски ће се чувати поруке и бити избацивана упозорења ако нешто није како треба (фали крајња позиција, почетна...). 
\\ \indent Други мод је обични мод и то је док корисник није сачувао полигон. У овом моду кад се кликне дугме назад, ништа неће бити сачувано. 
\subsection{Опције за уређивање полигона}
\begin{figure}[htb!]
\begin{center}
\includegraphics[scale=.1]{pictures/createPolygon/Basic}
\caption{Екран који корисник види кад уређује полигон}\label{fig:createPolygonBasic}
\end{center}
\end{figure}
Корисник има опције генерисања препрека за лопту наведеним у секцији \ref{UseCases:Game} кликтањем одговарајућег дугмета са именом. Такође може и да генерише почетну позицију лопте.  Поред тога са свим фигурама корисник може да ради једну од четири опције:
\begin{enumerate}
\item  \emph{Pomeri} - Помера одговарајућу фигуру на полигону тако да корисник мора да задржава прст на фигуи док је помера, Кад пусти, ту ће фигура бити померена. 
\item  \emph{Povecaj}- Повећава одговарајућу фигуру, тако да ако корисник одабере фигуру и помера прст, фигура ће се повећавати у односу на то где је његов прст (за круг ће крајња позиција прста означавати позицију тачке са кружнице, за правоугаоник позицију одговарајућег темена).
\item  \emph{Brisi} - Брише  одабрану фигуру са полигона.
\item  \emph{Rotiraj} - Ротира правоугаоник према позицији прста тако да одабрана права која је одређена центром правоугаоника и прстом ротира око центра пратећи прст, са том правом ротира и цео правоугаоник. 
\end{enumerate}

\subsection{Чување полигона}
\begin{figure}[htb!]
\begin{center}
\includegraphics[scale=.1]{pictures/createPolygon/Save}
\caption{Дијалог који се појави кад корисник хоће да сачува полигон}\label{fig:createPolygonSave}
\end{center}
\end{figure}
На крају корисник кад заврши генерисање правоугаоника има дугме \emph{SACUVAJ POLIGON}. Тад се кориснику појављује дијалог на коме уноси тежину полигона као и име тог полигона. Кад кликне \emph{SACUVAJ POLIGON} на новом дијалогу, сачуваће полигон. Ако је постојао стари пребрисаће га.


\section{Резултати}\label{UseCases:Statistics}
\begin{figure}[htb!]
\begin{center}
\includegraphics[scale=.1]{pictures/statistics/Choose}
\caption{Избор полигона за који хоће да се виде резултати}\label{fig:statisticChoose}
\end{center}
\end{figure}

\begin{figure}[htb!]
\begin{center}
\includegraphics[scale=.1]{pictures/statistics/HighScore}
\caption{Листа резултата за одговарајући полигон}\label{fig:statisticHighScore}
\end{center}
\end{figure}
Корисник има опцију да види статистику за све полигоне које је играо, и који постоје . Такође има моућност да види и за авантура мод (под именом \emph{Avantura}). Корисник кликом на спинер бира одговарајући ниво за који хоће да прикаже резултате (слика \ref{fig:statisticChoose}) и потом, му се приказују резултати сортирани од најбољег ка најлошијем (\ref{fig:statisticHighScore}) у погледу времена. Корисник има могућност ресетовања листе резултата за један изабрани полигон притискањем дугмета \emph{RESETUJ TRENUTNU} или ресетовањем резултата свих листи са притискањем дугмета \emph{RESETUJ SVE}.


\section{Подешавања}\label{UseCases:Settings}
\begin{figure}[htb!]
\begin{center}
\includegraphics[scale=.1]{pictures/settings/basic}
\caption{Подешавања за игру}\label{fig:settingsBasic}
\end{center}
\end{figure}
Корисник има опцију да подеси одговарајуће коефицијенте  које се користе током симулације игре (слика \ref{fig:settingsBasic}). То се постиже померањем одговарајуће тачке на дужи. Редом су наведени коефицијенти:
\begin{enumerate}
\item Коефицијент убрзања - Колико брзо ће лоптица да убрзава, што већи коефицијент брже убрзава.
\item Коефицијент губитка брзине - Колико енергије остаје у лоптици приликом судара, што већи коефицијент, више енергије остаје.
\item Коефицијент трења - Колико лоптица успорава приликом кретања по подози, што већи коефицијент више успорава.
\end{enumerate}
Корисник има могућност да врати подешавања свих коефицијената на подразумевана притиском на дугме \emph{RESET} из одговарајућег реда за коефицијенте. 




\chapter{Android}\label{Android}

\emph{Још од избацивања IPhone-а као првог комерцијалног "паметног" телефона који ће објединити у једном уређају све функционалности  које има PC (линк \cite{IPHONE}), ако не и више, било је јасно у ком смеру се запутило IT \footnote{ Technoloy} тржиште. Телефони или ти компјутери у џепу су у том тренутку постали будућност, а наша садашњост. Како ће се развијати IT технологија, није познато, али зна се да ће огроман удео имати телефони. Одатле и потиче моја  жеља да овладам вештином Android програмирања, оперативним системом који је преузео примат на тржишут паметних телефона. (линк \cite{MarkShare}) }
\section{GUI код}
Сав GUI који је коришћен, писан је у xml-у који подржва одговарајуће Android Studio IDE. Сва имена GUI компоненти давана су тако да прво иде тип компоненте и затим се надовезује одговарајући опис који је карактеристичан за употребу дате компоненте. Типа \emph{seekBarCoeficientAcceleration} означава \emph{seekBar} GUI компоненту, а CoeficientAcceleration каже да се користи да представи коефицијент убрзања. Имена сва су писана CamelCase-ом \footnote{\url{http://wiki.c2.com/?CamelCase}}. При избору компоненти биране су тако да што пријатније изгледа кориснику и да што лагодније буде за рад (на основу узорка од неколико корисника који су пробали различите верзије GUI-а). 
\\ \indent 
Где је било неопходно да позиционирање компоненти буде независно од типа од екрана коришћен је \emph{LinearLayout}
\footnote{\url{https://developer.android.com/reference/android/widget/LinearLayout.html}}, 
док је за неке ствари где је битна само позиција компоненти, коришћен \emph{RelativeLayout}\footnote{\url{https://developer.android.com/guide/topics/ui/layout/relative.html}}.
\\ \indent 
Коришћена је Android Dark Material Theme као основна тема.

\section{Java код}
GUI компоненте на одговарајућим екранима референциране су тако што се из одговарајућег прозора нађе компонента уз помоћ \emph{findViewById} \footnote{\url{https://developer.android.com/reference/android/view/View.html} и \url{https://developer.android.com/reference/android/app/Activity.html}}
\\ \indent 
Постоји пет активности (\emph{MainActivity, CreatePolygonActivity, GameActivity, SettingsActivity, StatisticsActivity}). Свака подржавајући одговарајућу функционалност из главе \ref{UseCases}.
\\ \indent 
Свака активност је прављена тако да је изведена из активности \emph{CommonActivtity}. Свака активност која наслеђује ову класу има подешен прозор тако да је навигациона трака скривена док корисник не превуче прстом са дна уређаја на горе. Поред тога оријентација је увек вертикално, да корисник не би губио време ако случајно окрене уређај. Такoђе екран обавештења је остао доступан кориснику да би могао у сваком тренутку да сазна више о обавештењу које му стигне (али тек након што превуче прстом екран са врха ка дну).
 Дата имплементације је по узору на већину данашањих екслузивних играчких наслова за Android уређаје попут Hill Climb Racing (линк за преузимање \cite{HillCR}). Даље коришћен је MVC пројектни узорак прилагођен за Android. При чему имамо активност која прослеђује своје догађаје контролеру, и он у зависности од њих обавља акције и у моделу се то чува. Постоји и имплементација где у моделу постоје методе које обрађују податке, али изабраним је раздвојена имплементација кода , од приступа подацима, и олакшана читкоћа кода. 
 \\ \indent
 Постојање класе \emph{CommonModel} за циљ има омогућавање заједничког модела свим активностима које су за потребу имали рендеровање направљеног/који се прави полигона. 
 \\ \indent
 Тамо где је разумно било да се појављују дијалози (као за чување резултата по успешној игри, или за чување направљеног полигона) прављене су класе које проширују класу 
 \emph{Dialog}\footnote{\url{https://developer.android.com/reference/android/app/Dialog.html}}. 
 Ово је омогућило финије подешавање дијалога и њиховог изгледа oд унапред направљених класа попут \emph{AlertDialog}\footnote{\url{https://developer.android.com/reference/android/app/AlertDialog.html}}.
\\ \indent 
Тамо где су била потребна исцртавања на екрану коришћене су одговарајућe класе које су проширивале \emph{SurfaceView} \footnote{\url{https://developer.android.com/reference/android/view/SurfaceView.html}} (биће касније објашњено како ово ради у глави \ref{Graphics}). 
\\ \indent 
За \emph{SettingsActivity} било је неопходно унапредити постојећу GUI компоненту SeekBar и додата је класа \emph{SeekBarUpgrade} која додаје још неки низ особина и омогућава лакше додавање више SeekBar-ова.

\section{Unit Test}
\chapter{Модел судара} \label{Collision}


\chapter{Графика} \label{Graphics}

\emph{Иако рачунарска графика на први поглед изгледа једноставна област, што се више задубљујеш у њу, откриваш каква озбиљна наука стоји. }

\section{Општа идеја}
Док су за активности у којима није потребно учестало исцртавања екрана коришћена GUI нит, тамо где је потребно (\emph{CreatePolygonActivity, GameActivity}) морало је бити пронађено другачије решење. За површ уместо стандардног \emph{Canvas}-a који припада \emph{ImageView}, који захтева исцртавање целог екрана \footnote{\url{https://developer.android.com/reference/android/widget/ImageView.html}} користи се \emph{SurfaceView} који се само освежи (остатак екрана се не мења) кад је неопходно. И то исцртавање се ради у посебној нити, која кад заврши посао, само замени \emph{Canvas} од SurfaceView са новим Canvas-ом. Што умањује заузетост GUI нити непотребним исцртавањем, и омогућава да се користи у рачунању и ажурирању SurfaceView.
Ради смањења загревања уређаја, нова нит која исцртава \emph{Canvas} то ради само кад је затражено од ње (кад се десила промена позиције), остатак времена спава. 
\\ \indent
 Цео модел је оптимизован тако да се тражило максималној паралелизацији и минимализацији броја lock-ова. У обе активности се на почетку иницијализације \emph{SurfaceView} правe класе \emph{ShapeFactory} и \emph{ShapeDraw}, од којих прва служи за парсирање полигона из фајлова (и њихово скалирање), док друга служи за цртање фигура по \emph{Canvas}-у \emph{SurfaceView}. Да би фигура била исцртана помоћу класе \emph{ShapeDraw} неопходno je да подржава \emph{ShapeDrawInterface}, тј. да може да се кликне на њу, ротира, промени величина, помери, израчуна угао нагиба. Такође при иницијализацији \emph{SurfaceView} прави се и посебна нит која ће да ради исцртавање. При уништавању \emph{SurfaceView} нит се уништава.
\section{Оптимизације код играња игре}
 Код играња игре, нема потребе за непотребно рендеровање и исцртавање других фигура по \emph{Canvas}-у осим на почетку. Стога се направи спрајт целог полигона без лопте, и лопта се лепи касније на спрајт како се мења њена позиција. Ово омогућава убрзано ажурирање екрана.
\chapter{Aрхитектура} \label{Architecture}

\section{Организација пакета}

\begin{figure}[htb!]
\begin{center}
\includegraphics[scale=.7]{pictures/UML/package/application}
\caption{Организација класа по пакетима (application пакет)}\label{fig:umlPackageApp}
\end{center}
\end{figure}

\begin{figure}[htb!]
\begin{center}
\includegraphics[scale=.6]{pictures/UML/package/logic}
\caption{Организација класа по пакетима (logic пакет)}\label{fig:umlPackageLog}
\end{center}
\end{figure}

Читав java код је организован тако да се налази унутар пакета \emph{com.example.popina.projekat} и то у два потпакета. Код који је везан за MVC преглед и Android део налази се у applicatiоn потпакету (слика \ref{fig:umlPackageApp}). Код који је везан за логику игре (база података, како се праве фигуре, парсира...) налази се у потпакету logic (\ref{fig:umlPackageLog}). Логика класа из потпакета application је објашњена у глави \ref{Android}, као и у глави \ref{Graphics}. Такође логика пакета \emph{coefficent} je објашњена у глави  \ref{Android}, као и логика \emph{collision} пакета унутар кога је \emph{collisionMode}l. Стога овде ће бити објашњена организација кода у потпакету logic који описује логику тј. како ради апликација. 

\section{Пакет game}
\subsection{Пакет acceleration.filter}

\begin{figure}[htb!]
\begin{center}
\includegraphics[scale=.6]{pictures/UML/class/filter}
\caption{Класа filter}\label{fig:umlClassFilter}
\end{center}
\end{figure}

Унутар овог пакета се налази интерфејс \emph{FilterInterface} чија метода \emph{filter} прима очитане вредности убрзања са улаза и враћа филтриране вредности. Ово спречава да се дешавају нагле промене убрзања услед случајно лошег очитавања сензора. У апликацији је имплементирана у виду \emph{FilterPastValue} класе (слика \ref{fig:umlClassFilter}).Ова класа филтрира тако што последњу филтрирану вредност и ону детектовану скалира тако да $\alpha$($0 \leq \alpha \leq 1$) се множи са новом вредношћу, а са $1-\alpha$ са старом и то се сабира. Ова класа се користи код филтрирања вредности убрзања у \emph{GameActivity}. 

\subsection{Пакет coefficient}
У овом пакету класа \emph{Coefficient} са методом \emph{updateValues} чува коефицијенте унутар \emph{SharedPreferences} (чије име се налази у \emph{CoefficientModel}). 
\subsection{Пакет collision}
Класа \emph{CollisionModelAbstract} има методу \emph{updateSystem} која прима за аргументе филтрирано убрзање. време детекције сензора, листу  \emph{Figure} којe представљају препреке, или циљ за лопту, као и саму лопту. Након позива ове методе треба да буде ажурирана позиција лопте, као и пуштен одговарајући звук player-ом који је примила класа у конструктору. Враћа једну од 4 вредности које кажу да ли је игра побеђена, изгубљена, да ли има колизије или нема колизије. Метода \emph{setLastTime} служи за иницијализцију референтног времна од кога ће мерити промена брзине лопте.
\subsection{Пакет utility}

Овде се садрже како само име каже Utility ствари као што су \emph{Coordinate3D} која представља 3D координату тачке у простору. \emph{Time} која представља временски интервал која има почетак и крај и чија дужина може да се рачуна преко методе \emph{timeInt}.  
\\ \indent 
Класа \emph{Utility} обухвата методе које служе за рад са координатама, конверзијама, насумичним бројем. Редом су наведени потиси и описи:
\begin{itemize}
\item \emph{static double radianToDeg(float rad)} - претвара угао \emph{rad} из радијана у степене и враћа као повратну вредност.
\item \emph{static doubledegToRadian(float deg)} -  претвара угао \emph{deg} у степенима у радијане и враћа као повратну вредност.
\item \emph{static float convertMsToS(float ms)} - претвара из милисекунди \emph{ms} у вредност у секундама. 
\item \emph{static float convertNsToS(float ns)} - претвара из наносекунди \emph{ns} у вредност у секундама.
\item \emph{static float opositeSign(float val)} - враћа супротан знак од броја \emph{val}.
\item \emph{static float convertRadianAngleTo2PiRange(float angle)} - пребацује угао \emph{angle} у радијанима у $\left[ o, 2 \pi \right]$ интервал.
\item \emph{static double randomNumberInInterval(int startInterval, int endInterval)} / враћа број у задатом интервалу \emph{[startInterval, endinterval]}.
\item \emph{static Coordinate rotatePointAroundCenter(...)}-ротира тачку око центра за одређени угао (тамо где нема центра узима се $(0, 0)$ за центар, тамо где нема угла користе се израчунате вредности синуса и косинуса које су прослеђене за брже рачунање ротације).
\item \emph{static float calculateAngle(Coordinate center, Coordinate point)} -  рачуна угао између $x$ осе и праве одређене тачком  \emph{point}и центром \emph{center}.
\item \emph{static boolean doesSegmentIntersectsCircle(Coordinate beginSegment, Coordinate endSegment, Coordinate center, float radius, boolean isXLine)} - одређује да ли круг (\emph{Coordinate} центар и \emph{radius} полупречник) сече прослеђени сегмент (почетак сегмента је тачка \emph{beginSegment}, крај \emph{endSegment}), са тим да су сегменти увек паралелни $x$ или $y$ оси што се прослеђује параметром \emph{isXLine} (да ли је $x$ оса). 
\item \emph{static boolean isDimBetweenDims(float dimBegin, float dimEnd, float dim)} -  одређује да ли је вредност \emph{dim} између две вредности \emph{dimBegin} и \emph{dimEnd} на реалној правој, при чему се користи одступање од $0,01$.
\item \emph{static float distanceSquared(Coordinate point1, Coordinate point2)} - враћа растојање између две координате \emph{point1 \emph{и} point2}квадрирано.
\item \emph{ public static boolean isDistanceBetweenCoordLesThan(Coordinate coordinate1, Coordinate coordinate2, float dist, boolean isSquared)} -  Одређује да ли је растојање између две координате \emph{coordinate1 \emph{и} coordinate2} мање од прослеђеног \emph{dist}, при чему се користи тачност од $0,01$. Aко јe растојање већ квадрирано нема потребе да се квадрира (што се може проследити као параметар \emph{isSquared}).
\end{itemize}

\section{Пакет statistics.database}

Овај пакет садржи две класе \emph{GameDatabaseHelper} и \emph{GameDatabase}.  \emph{GameDatabaseHelper} проширује класу \emph{SQLiteOpenHelper} \footnote{\url{https://developer.android.com/reference/android/database/sqlite/SQLiteOpenHelper.html}} и служи да омогући згодније мењање шема база (уништавање старе базе и ажурирање на нову).
\\ \indent
 \emph{GameDatabase} је једна огромна фасада за приступ бази податка. 
Редом су наведене њене методе као и њени описи:
\begin{itemize}
\item \emph{String getFirstLevel()} - враћа име првог полигона из табеле \emph{LevelTable}.
\item \emph{int insertUser(String user, String levelName, long time)} - убацује време \emph{time}корисника \emph{user} за полигон \emph{levelName} у табелу \emph{UserScoreTable}.
\item \emph{int insertLevel(String levelName, int levelDifficulty)} - убацује полигон \emph{levelName}у табелу \emph{LevelTable} при чему му је тежина \emph{levelDifficulty}. У случају да постоји полигон са истим именом стари полигон ће бити обрисан.
\item \emph{Cursor queryHighScore(String levelName)} - враћа \emph{Cursor}\footnote{\url{https://developer.android.com/reference/android/database/Cursor.html}} који кад се итерира садржи сортирану опадајуће листу времена са одговарајућим корисницима за полигон \emph{levelName}.
\item \emph{int deleteHighScore(String level)} - брише листу времена за полигон \emph{level} из табеле \emph{UserScoreTable}.
\item \emph{int deleteLevel(String level)} - брише полигон (а самим тим и резултате везане за њега) из базе података-
\item \emph{int getDifficulty(String levelName)} - враћа тежину за одговарајући полигон \emph{levelName}
\item \emph{LinkedList<String> getLevels(int difficulty)} - враћа листу  нивоа са тежином \emph{difficulty}.
\end{itemize}

\subsection{Пакет table}

Овај пакет садржи две класе које у суштини представљају табеле, тј. садрже имена колона која им припадају као и одговарајући SQL који служи за њихово генерисање и уништавање. 
\\ \indent
 \emph{LevelTable} у себи садржи поред \emph{\_ID}-а и колоне \emph{TABLE\_COLUMN\_LEVEL\_NAME } (представља име полигона које корисник сачува) као и \emph{TABLE\_COLUMN\_LEVEL\_DIFFICULTY} што представља тежину нивоа. Стављен је \emph{Unique constraint} ограничење на име полигона, да случајно се не дода више редова са истим именом полигона, него да се увек ажурира један. \
\\ \indent 	
 emph{UserScoreTable } у себи садржи поред \emph{\_ID}-а и колоне \emph{TABLE\_COLUMN\_USER\_NAME } што представља име играча који је на ранг листи, Ту су и колона \emph{TABLE\_COLUMN\_TIME}  која представља време за које је пређен тај полигон. Kao и колона \emph{TABLE\_COLUMN\_FK\_LEVEL} која садржи \emph{\_ID} реда из табеле \emph{LevelTable} који референцира (тј. полигон ком припада тај резултат).
 
\section{Пакет shape}
\subsection{Пакет constants}
Унутар овог пакета се налазе две класе \emph{ColorConst} и \emph{ShapeConst}. \emph{ColorConst} чува константе везане за боје одређених фигура. Док \emph{ShapeConst}  чува константе везане за почетне позиције одређених фигура и величину. Kao и податке неопходне при чувању полигона у фајл (имена фигура, којим редом се стављају параметри...)
\subsection{Пакет coordinate}
Унутар овог пакета налази се класа \emph{Coordinate} која представља једну тачку или вектор (у зависности од тога за шта је потребна). Стога подрава скларани производ два вектора (\emph{scalarProduct}), одузимање (\emph{subCoordinate}) и сабирање вектора (\emph{addCoordinate} - враћа као нови вектор, док \emph{addToThisCoordinate} мења вектор за који је позвана),величину (\emph{magnitudeSquared}) и дохватање и мењање одговарајућих координата. Такође подржава методу \emph{toString} јер се користи код чувања полигона. 
\subsection{Пакет draw}
Има једну класу и један интеррфејс. 
Интерфејс \emph{ShapeDrawInterface} има следеће уговоре које класе које га имплементирају морају имати :
\begin{itemize}
\item \emph{void drawOnCanvas(Canvas canvas)} - мора да црта себе на \emph{canvas} -у.
\item \emph{void moveTo(Coordinate coordinate)} - мора да помери своју позицију (центар или како је везано) на координату \emph{coordinate}.
\item \emph{void resize(Coordinate c)} - мора да промени величину ако се зна да је кликнута координата \emph{c}.
\item \emph{boolean isCoordinateInside(Coordinate c)}- враћа \emph{true} ако је \emph{c} унутар дате фигуре.
\item \emph{void rotate(Coordinate c, float angle)}  - ротира фигуру на кликнуту тачку \emph{c}, ако се зна почетни угао нагиба фигуре  \emph{angle} кад је кренуло ротирање фигуре.  
\item \emph{float calculateAngle(Coordinate point)}  - рачуна угао измећу угла ротиране фигуре и оног одређеног центром те фигуре и тачком \emph{point}
\end{itemize}

Класа \emph{ShapeDraw} служи за исцртавање \emph{ShapeDrawInterface} по \emph{Canvas}-у. Њена употреба може се прочитати у глави \ref{Graphics}. Поље типа \emph{CommonModel} служи за синхронизацију међу нитима. Методе које подржава:
\begin{itemize}
\item \emph{void spriteOnBackground(LinkedList<? extends ShapeDrawInterface> listFigures)} - црта листу фигура \emph{listFigures} на \emph{Canvas} који већ садржи..
\item \emph{void drawOnCanvas(LinkedList<? extends ShapeDrawInterface> listFigures, Canvas canvas)} -црта листу фигура \emph{listFigures}на \emph{canvas}, при чему се прво постави подразумевана позадина која је учитана као позадина (по њој ће се цритати).
\item \emph{public void drawOnCanvas(ShapeDrawInterface shapeDrawInterface, Canvas canvas)} - црта фигуру \emph{shapeDrawInterface }на \emph{canvas}.
\end{itemize}

\subsection{Пакет factory}
Овај пакет представља модификацију пројектног обрасца Апстрактна фабрика. При чему не постоји више фабрика, него једна која прави све објекте и која прима \emph{UtillScale} за скалирање фигура (кад се прочитају из фајла у процентима, да би направила онакве какви одговарају екрану корисника), као и за обрнуто скалирање (кад треба да се сачувају). Њена основна намена је прављење фигура одговарајућег типа. Клас којом је она подржана је \emph{ShapeFactory}. Следе битне методе:
\begin{itemize}
\item \emph{StartHole createStartHole()}-прави почетну позицију лопте, скалирану за екран уређаја.
\item \emph{FinishHole createFinishHole()}- прави рупу у коју лопта треба да уђе, скалирану за екран уређаја.
\item \emph{WrongHole createWrongHole()}- прави амбис у који лопта не сме да упадне, скалиран за уређај екрана.
\item \emph{RectangleObstacle createObstacleRectangle()}- прави правоугаону препреку од које се лопта одбија, скалирану за уређај екрана.
\item \emph{CircleObstacle createObstacleCircle()}- прави кружну препреку од које се одбија лопта, скалирану за уређај екрана.
\item \emph{VortexHole createVortexHole()}-прави вртлог у који лопта ако упадне врти се и избацује из ње, скалиран за уређај екрана.
\item \emph{Figure scaleFigure(Figure f)}- скалира фигуру \emph{f} за екран користећи \emph{UtilScale}који је прослеђен у конструктору .
\item \emph{Figure scaleReverse(UtilScale utilScale)}- скалира фигуру \emph{f} за фајл користећи \emph{UtilScale}који је прослеђен у конструктору .
\item \emph{LinkedList<Figure> scaleFigures(LinkedList<Figure> listFigures)}-скалира листу фигура \emph{listFigurs}за екран користећи \emph{UtilScale}који је прослеђен у конструктору.
\item \emph{inkedList<Figure> scaleReverseFigures(LinkedList<Figure> listFigures)}-скалира листу фигура \emph{listFigures}за фајл користећи \emph{UtilScale}који је прослеђен у конструктору.
\end{itemize}
Класа \emph{ShapeBorderFactory} садржи методу чији потпис је \\\emph{LinkedList<RectangleObstacle> createBorders()} и која генерише листу зидова (четири) као листу правоугаоних препрека скалираних за екран. Ово омогућава избегавање посебних провера за зидове (јер се зидови посматрају као фигуре).

\subsection{Пакет figure}
Унутар овог пакета, а и његових подпакета налазе се класе које имплемнтирају лопту, и препреке. 

\emph{Figure} класа имплементира интерфејсе \emph{ShapeParserInterface, ShapeDrawInterface, SoundInterface, CollisionDetectionInterface} који се користе код чувања у фајлове, пуштања звука, исцртавања, и детектовања и обрађивања колизије. Поред тога ако фигура дође у стање мировања омогућава да се звук не пушта више. Звук се такође не пушта ако лопта има учесталу колизију са препрекама. Такође \emph{toString} метода је неопходна због чувања у фајловима. Има  свој центар који ће се користити код померања, ротирања... Центар може да се дохвати и постави.


\subsubsection{Пакет hole}
Постоји класа \emph{CircleHole} која имплемнтира лопту и лоптасте препреке у 2D. Изведена је из класе \emph{Figure}.Самим тим имплементира методе из \emph{ShapeDrawInterface} и \emph{CollisionDetectionInterface}, уз додато поље полупречник, које може да се дохвати и постави.
Постоји и класа \emph{StartHole} која служи за приказивање лопте, додатно још имплементира методе из \emph{SoundInterface} као и промењену \emph{toString} методу. 
\\ \indent Садржи  пакет gravity који служи за класе које представљају рупе (тј. има неки облик гравитације који вуче лопту ка њима). Стога \emph{ GravityHole} имплементира још \emph{CollisionHandlingInterface} који омогућава промену брзине након судара (у суштини мења брзину тако да иде ка центру лопте). У случају \emph{FinishHole} и \emph{WrongHole} које прошиују \emph{GravityHole} долази до краја игре кад лопта упадне у њих (с тим да је у једном случају позитиван, а у другом негативан крај). Док \emph{VortexHole} такође проширује, али игра се не губи кад лопта упадне у њу, нити побеђује. Такође \emph{toString} и звук су другачји за сваки тип препреке.
\subsubsection{Пакет obstacle}
У овом пакету постоје класе \emph{Obstacle}, \emph{CircleObstacle} и \emph{RectangleObstacle}. Прва проширује класу \emph{Figure}, и поставља одговарајући звук који ће бити пуштан за препреке. Такође имплементира \emph{CollisionHandlingInterface} који се користи након судара да се промени брзина објекта, тј. није крај игре кад се лопта судари са њима. Друге две класе проширују класу \emph{Оbstacle}. И код \emph{CircleObstacle} и \emph{RectangleObstacle} све методе класе \emph{Figure} су override-оване и самим тим ту су имплементације за фигуре типа правоугаоник и круг у 2D. Поред тога \emph{RectangleObstacle} садржи и поља \emph{width} и \emph{height} која могу да се дохвате и поставе и представљају одговарајућу ширину и висину. Може да се дохвате и координате одговарајућих темена. Слично за лопту може да се дохвати полупречник.

\subsection{Пакет movement.collision.detection}
Налази се интерфејсе \emph{CollisionDetectionInterface}. Класе које га имплементирају морају да имају следеће методе:
\begin{itemize}
\item \emph{boolean doesCollide(CircleHole ball)} - да ли постоји судар између лопте \emph{ball} и објекта класе која имплементира интерфејс .
\item \emph{boolean isGameOver()} - да ли је игра по судару лопте са  објектом класе која имплементира интерфејс готова.
\item \emph{boolean isWon()} - да ли је игра по судару лопте са  објектом класе  која имплементира интерфејс побеђена или изгубљена (мора прво претходна метода да врати \emph{true}).
\end{itemize}
\subsection{Пакет movement.collision.handling}
Налази се интерфејсе \emph{CollisionHandlingInterface}. Класе које га имплементирају морају да имају следеће методе:
\begin{itemize}
\item \emph{Coordinate getSpeedChangeAfterCollision(StartHole ballOld, StartHole ballNew, Coordinate3D speed)} - враћа вектор који треба додати вектору брзине \emph{speed}тренутног кретања лопте \emph{ballOld}, при чему је потенцијална нова позиција \emph{ballNew}. 
\end{itemize}

\subsection{Пакет parser}
	
Садржи класе и интерфејсе који омогућавају чување полигона у фајл и његово учитавање ради даљег приказивања на екран.

\emph{ShapeParserInterface} је интерфејс који омогућава читање фигура из фајлова и њихово скалирање. Следеће методе класе које имплементирају интерфејс морају имати:
\begin{itemize}
\item \emph{ShapeParserInterface scale(UtilScale utilScale)} - враћа објекат класе која имплементира \emph{ShapeParserInterface} скалирану за екран уз помоћ \emph{utilScale}.
\item \emph{ShapeParserInterface scaleReverse(UtilScale utilScale)} - враћа објекат класе која имплементира \emph{ShapeParserInterface} скалирану за фајл (у процентима) уз помоћ \emph{utilScale}.
\end{itemize}

\emph{ShapeParserAbstract} је апстрактна класа која чита \emph{ShapeParserInterface} из фајла и уз помоћ објекта класе \emph{ShapeFactory} прави фигуре, и уз помоћ објекта класе \emph{ShapeDraw} их исцртава на екрану. Класе које је изводе морају да подрже следеће методе:
\begin{itemize}
\item \emph{ShapeParserInterface parseLine(String line)}- изведена класа парсира линију фајла и враћа објекат интерфејса\emph{ShapeParserInterface}.
\item \emph{LinkedList<? extends ShapeParserInterface> parseFile(String fileName)}-парсира фајл \emph{fileName}при чему позива \emph{parseLine} методу, и генерише листу фиугра спремних за цртање на дати екран.
\item \emph{void drawImageFromFile(Canvas canvas, String fileName)}- црта фигуре из фајла \emph{fileName} на платно \emph{canvas}.
\end{itemize}

Класа \emph{ShapeParser} је изведена из \emph{ShapeParserAbstract} и подржава све методе из њеног интерфејса, с тим да свуда генерише објекте изведених класа изq \emph{Figure} уместо \emph{ShapeParserInterface}

\subsection{Пакет scale}
Поседује две класе. Прва је апстрактна \emph{UtilScale} и кад се иницијализује прима прима величину екрана и на основу тога скалира дужине по висини (\emph{scaleHeight} и \emph{scaleReverseHeight}) ширини (\emph{scaleWidth} и \emph{scaleReverseWidth}) као и координате (\emph{scaleReverseCoordinate} и \emph{scaleCoordinate}). \emph{scalexxx} скалира вредности за екран уређаја, док \emph{scaleReverse} скалира за фајл (претвара у проценте). Класа \emph{UtilScaleNormal}импементира претходно наведене методе.
\subsection{Пакет sound}
Садржи класе и интерфејсе за звук током судара лопте са препреком. Класа \emph{SoundConst} садржи  константе везане за имплементацију музичког player-а. \emph{SoundPlayerCallback} је интерфејс чије имплементације омогућавају пуштање звука са редним бројем дефинисаним у \emph{SoundConst} (као \emph{idSound} се прослеђује) преко методе чији је потпис 
\\  \indent \emph{void playSound(int idSound)}.\\  Имплементиран је у виду класе \emph{SoundPlayer}.
\\ \indent
\emph{SoundInterface} интерфејс условљава фигуре којега имплементирају да пуштају одређени звук приликом судара преко методе\\ \indent  \emph{void playSound(SoundPlayerCallback soundPlayerCallback)}.\\
При чему је \emph{soundPlayerCallback} player преко кога се пушта звук.

 
 
\chapter{Закључак} \label{Conclusion}

Игрица је за циљ имала да научим Android API. Али проблеми попут рачунарске графике и моделирања физике пребацили су тежиште пројекта.  
\\ \indent Стога игрица је постигла првобитни циљ, а то је Android апликација чији код је читак, добро организован, лак за проширивање. Додавање нових препрека у игри је веома једноставно извођењем из постојећих класа. Додавање нових опција у модел физике је веома једноставно мењањем једне методе. Механика играња тј. исцртавња фигура апликације је оптимизована до границе кад корисник не осећа да се ради о Android уређају. У суштини, читав апликација је добро организована и проширива у свим погледима. 
\\ \indent Међутим проблем савршене физике остаје отворено питање. Иако ми физика није била омиљени предмет током школовања, модел који је урађен, довољан је да корисник не примети несавршеност физике у неким деловима. Требало да се среди даљим изучавањем моделирања физике у оваквим типовима апликације. 
\\ \indent Следеће отворено питање да ли користити OpenGL и да ли ће он дати још већа побољшања у погледу исцртавања фигура. И са тим питање, да ли треба да се поред лопте која се креће треба да дода неколико покретних препрека. Да ли би оне кориснику дале још више уживања? Све ово захтева темељно испитивање OpenGL API-а за Anrdoid и његово темељно тестирање на одговарајућим моделима физике који буду коришћени.
\\ \indent Треће отворено питање је изглед апликације. Иако сам се трудио у сваком погледу да постигнем да апликација буде што привлачнија за кориснике, видно је да неки делови захтевају побољшање (попут боја препрека, лопте, могућност постављања позадине нивоа, ...).  Стога треба наћи добро GUI дизајнера за Android који би предложио неке промене. 
\\ \indent Последње отворено питање је сама логика апликације, тј. како се ради adventure мод игрице, и да ли треба дозволити кориснику да прави сам нивое. Стога треба извршити истраживање код корисника да се види шта они очекују од овакве апликације по питању нивоа.
\\ \indent И последње питање је, да ли апликација треба да иде у продукцију, тј. на Google Play Store (линк \cite{GooglePlayStore}). Ако буде постојала жеља, неопходно је систем тестирања подићи на веома озбиљан ниво, да се и најмањи багови уклоне.

\addcontentsline{toc}{chapter}{\bibname}



\begin{thebibliography}{refs}

		\bibitem{PMU}
		Саша Стојановић, Захарије Радивојевић, Милош Цветановић.
		\emph{Програмирање мобилних уређаја}.\\
		\url{http://rti.etf.bg.ac.rs/rti/si4pmu/}


	\bibitem{AndroidDeveloper}
		Google.
		\emph{Android Developers}.\\
		\url{https://developer.android.com/index.html}


		\bibitem{StackOverflow}
		Stack Exchange Inc.
		\emph{stack overflow}.\\
		\url{https://stackoverflow.com/}

		\bibitem{GTA}
		RockStar Games.
		\emph{GrandTheftAuto}.
		\url{http://www.rockstargames.com/grandtheftauto/}
		
			\bibitem{BitBucket}
		Ђорђе Живановић.
		\emph{Ball Game}.
		\url{https://bitbucket.org/popina1994/ball-game}		
		
		\bibitem{EngBook}
		Ian Millington.
		\emph{Game Physics Engine Development}.
		CRC Press.
		2nd Edition. 2010.




		\bibitem{ModCol}
		Ian Millington.
		\emph{Cyclone physics system}.
		\url{https://github.com/idmillington/cyclone-physics/tree/master/src}

		\bibitem{HillCR}
		Fingersoft.
		\emph{Hill Climb Racing}.
		\url{https://play.google.com/store/apps/details?id=com.fingersoft.hillclimb&hl=sr}

		\bibitem{GooglePlayStore}
		Google.
		\emph{Google Play}.
		\url{https://play.google.com/store}






\end{thebibliography}

Свим линковима је приступано последњи пут \today


\end{document}